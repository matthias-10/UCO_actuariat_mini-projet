\documentclass[a4paper,12pt]{scrartcl}

\usepackage[T1]{fontenc}
\usepackage{lmodern}
\usepackage[utf8]{inputenc}
\usepackage[francais]{babel}
\usepackage{graphicx}
\usepackage{microtype} 
\usepackage[onehalfspacing]{setspace}
\usepackage[top=2.5cm, bottom=2.5cm, left=3cm, right=3.5cm]{geometry}
\usepackage{tabularx}
\usepackage{graphicx}
\usepackage{longtable}
\usepackage{listings}
\usepackage{listingsutf8}
\usepackage{dsfont}
\usepackage{appendix}
\usepackage{mathtools, amssymb}
\usepackage[usenames,dvipsnames]{color}


\definecolor{MyDarkGreen}{rgb}{0.0,0.4,0.0}
\lstloadlanguages{Matlab}
\lstset{language=Matlab,                        % Use MATLAB
        frame=single,                           % Single frame around code
        basicstyle=\small\ttfamily,             % Use small true type font
        keywordstyle=[1]\color{Blue}\bfseries,  % MATLAB functions bold and blue
        keywordstyle=[2]\color{Purple},         % MATLAB function arguments purple
        keywordstyle=[3]\color{Blue}\underbar,  % User functions underlined and blue
        identifierstyle=,                       % Nothing special about identifiers
                                                % Comments small dark green courier
        commentstyle=\usefont{T1}{pcr}{m}{sl}\color{MyDarkGreen}\small,
        stringstyle=\color{Purple},             % Strings are purple
        showstringspaces=false,                 % Don't put marks in string spaces
        tabsize=5,                              % 5 spaces per tab
        %
        %%% Put standard MATLAB functions not included in the default
        %%% language here
        morekeywords={xlim,ylim,var,alpha,factorial,poissrnd,normpdf,normcdf},
        %
        %%% Put MATLAB function parameters here
        morekeywords=[2]{on, off, interp},
        %
        %%% Put user defined functions here
        morekeywords=[3]{brownmo},
        %
        morecomment=[l][\color{Blue}]{...},     % Line continuation (...) like blue comment
        numbers=left,                           % Line numbers on left
        firstnumber=1,                          % Line numbers start with line 1
        numberstyle=\tiny\color{Blue},          % Line numbers are blue
        stepnumber=5,                           % Line numbers go in steps of 5
        literate=%                              % accents and Umlaute
                 {Ö}{{\"O}}1
                 {Ä}{{\"A}}1
                 {Ü}{{\"U}}1
                 {ß}{{\ss}}1
                 {ü}{{\"u}}1
                 {ä}{{\"a}}1
                 {ö}{{\"o}}1
                 {\$}{{\dollar}}1
        }
% Includes a MATLAB script.
% The first parameter is the label, which also is the name of the script
%   without the .m.
% The second parameter is the optional caption.

\title{Mini projet 1: Calcul du prix d'une option asiatique}
\author{Valentin DE CRESPIN DE BILLY \\ Matthias LANG}
\date{30.11.2021}

\linespread{1.5} 


\begin{document}

\maketitle
\begin{center}

  \thispagestyle{empty}

  N. d'étudiant: XXXXXXX et 313411\\
  Université Catholique de l'Ouest\\
  Mathématiques financières

\end{center}

\newpage

\section{Calculer le prix du sous-jacent}

\begin{equation} \label{1} 
dS_t~=~S_t(rdt+\sigma \sqrt{S_t} dW_t) 
\end{equation}

\begin{equation} \label{2}
\begin{multlined}
     \iff \frac{dS_t}{S_t} = rdt+\sigma \sqrt{S_t} dW_t
\end{multlined}
\end{equation}


On prend l'équation 1:
\begin{equation} \label{3}
\begin{multlined}
= dS_t~=~S_trdt+\sigma S_t^{1.5} dW_t \\
\text{Puis} \\
d \langle S_t,~~ S_t\rangle 
=\langle dS_t,~~ dS_t\rangle ~=\\
=\langle S_trdt+\sigma S_t^{1.5} dW_t,~~
         S_trdt+\sigma S_t^{1.5} dW_t \rangle ~=\\
=\langle \sigma S_t^{1.5} dW_t,~~
         \sigma S_t^{1.5} dW_t \rangle ~=\\
=S_t^3 \sigma^2 \langle dW_t,~~ dW_t \rangle ~=\\
=S_t^3 \sigma^2 dt
\end{multlined}
\end{equation}



\begin{equation} \label{4}
\begin{multlined}
\text{On pose: } X_t = ln(S_t) \\
\text{Formule d'Ito: } dln(S_t) = \frac{dS_t}{S_t} + \frac{1}{2} \frac{-1}{S_t^2}d \langle S_t, ~S_t \rangle \\
\text{Avec les équations 2 et 3:} \\
dln(S_t) = rdt + \sigma \sqrt{S_t} dW_t - \frac{1}{2}S_t \sigma^2 dt ~= \\
= (r - \frac{1}{2}S_t\sigma^2)dt + \sigma\sqrt{S_t}dW_t
\end{multlined}
\end{equation}


\begin{equation} \label{5}
\begin{multlined}
ln( \frac{S_t}{S_0} ) = ln(S_t)-ln(S_0) = \int_0^t dln(S_u) = \\
= \int_0^t (r-\frac{1}{2} S_t \sigma^2)du~+~\int_0^t \sigma \sqrt{S_t}dW_t \\
\dots
\end{multlined}
\end{equation}

Donc on ne peut pas facilement dériver une formule pour le prix comme ça, qui dépend que des variables fixées, mais on peut le simuler pas à pas avec (1):

\begin{equation} \label{6}
\begin{multlined}
S_0 ~\text{soit connu} \\
dS_0 = S_0(rdt + \sigma \sqrt{S_0} dW_0) \\
S_1 \approx S_0 + dS_0 \\
dS_1 = S_1(rdt + \sigma \sqrt{S_1} dW_1) \\
S_2 \approx S_1 + dS_1 \\
\dots
\end{multlined}
\end{equation}







%%%%%%%% A N N E X E %%%%%%%%%%%%%%%%%%
\appendix
\appendixpage
\addappheadtotoc

\section{Code Matlab}
\lstinputlisting{mini_projet.m}

\section{Code VBA}
\begin{lstlisting}[
    breaklines=true,
    tabsize=3,
    showstringspaces=false
    extendedchars=\true,
    language={[Visual]Basic},
    frame=single,
    framesep=3pt,%expand outward.
    framerule=0.4pt,%expand outward.
    xleftmargin=3.4pt,%make the frame fits in the text area. 
    xrightmargin=3.4pt,%make the frame fits in the text area.
    ]


Rem Attribute VBA_ModuleType=VBAModule
Option VBASupport 1
Sub Macro1()

Dim T, n, nt, Nd As Integer
Dim r, sigma, S0, t0 As Double

Dim i, j As Integer


r = Range("A2").Value
sigma = Range("A3").Value
T = Range("A4").Value
n = Range("A5").Value
nt = Range("A6").Value
Nd = Range("A7").Value
S0 = Range("A8").Value
t0 = Range("A9").Value

Dim dt As Double
dt = ((T - t0) / n)

Dim temps() As Double
ReDim temps(n + 1)
temps(0) = t0
For j = 1 To n + 1
    temps(j) = temps(j - 1) + dt
Next


Range("I2:I" & UBound(temps) + 1) = WorksheetFunction.Transpose(temps)

Dim S() As Double
ReDim S(1 To n + 1, 1 To nt)
Dim W As Double
Dim x As Double

For j = 0 To nt - 1
For i = 0 To n

  If i = 0 Then
    W = 0
    Else
    W = W + Sqr(-2 * Log(Rnd())) * Cos(6.283185307 * Rnd()) * Sqr(dt)
  End If
  
  x = S0 * Exp((r - (sigma ^ 2) / 2) * (temps(i) - t0) + sigma * W)
  Cells(2 + i, 10 + j).Value = x ' copier s dans la worksheet
  'S(i, j) = x

Next
Next



'Cells(i, 3).Value =


'S = zeros(length(t),nt);
'for i = 1:nt
'    S(:,i) = brownmo(S0, r, sigma ,t0, T, n); %brownmo est definie en bas
'End

'function S = brownmo(X0, mu, sigma, t0, T, n) %x0

'  delta = (T-t0)/n;
'  W = zeros(1,n+1);
'  tseq = t0:((T-t0)/n):T;
'  for i = 2:(n+1)
'    W(i) = W(i-1)+normrnd(0,1)*sqrt(delta);
'  End
'  S = X0 * exp( (mu-(sigma^2)/2) * (tseq-t0) + sigma*W );

MsgBox "Simule pour " & nt & " trajectoires."

'Sheets("Dashboard").Activate
'Range("Parametres").Select
'Range("A13").Value = T

End Sub
\end{lstlisting}



\end{document}
